% Options for packages loaded elsewhere
\PassOptionsToPackage{unicode}{hyperref}
\PassOptionsToPackage{hyphens}{url}
%
\documentclass[
  11pt,
  ignorenonframetext,
]{beamer}
\usepackage{pgfpages}
\setbeamertemplate{caption}[numbered]
\setbeamertemplate{caption label separator}{: }
\setbeamercolor{caption name}{fg=normal text.fg}
\beamertemplatenavigationsymbolsempty
% Prevent slide breaks in the middle of a paragraph
\widowpenalties 1 10000
\raggedbottom
\setbeamertemplate{part page}{
  \centering
  \begin{beamercolorbox}[sep=16pt,center]{part title}
    \usebeamerfont{part title}\insertpart\par
  \end{beamercolorbox}
}
\setbeamertemplate{section page}{
  \centering
  \begin{beamercolorbox}[sep=12pt,center]{part title}
    \usebeamerfont{section title}\insertsection\par
  \end{beamercolorbox}
}
\setbeamertemplate{subsection page}{
  \centering
  \begin{beamercolorbox}[sep=8pt,center]{part title}
    \usebeamerfont{subsection title}\insertsubsection\par
  \end{beamercolorbox}
}
\AtBeginPart{
  \frame{\partpage}
}
\AtBeginSection{
  \ifbibliography
  \else
    \frame{\sectionpage}
  \fi
}
\AtBeginSubsection{
  \frame{\subsectionpage}
}
\usepackage{amsmath,amssymb}
\usepackage{iftex}
\ifPDFTeX
  \usepackage[T1]{fontenc}
  \usepackage[utf8]{inputenc}
  \usepackage{textcomp} % provide euro and other symbols
\else % if luatex or xetex
  \usepackage{unicode-math} % this also loads fontspec
  \defaultfontfeatures{Scale=MatchLowercase}
  \defaultfontfeatures[\rmfamily]{Ligatures=TeX,Scale=1}
\fi
\usepackage{lmodern}
\usetheme[]{Madrid}
\usecolortheme{beaver}
\usefonttheme{structurebold}
\ifPDFTeX\else
  % xetex/luatex font selection
\fi
% Use upquote if available, for straight quotes in verbatim environments
\IfFileExists{upquote.sty}{\usepackage{upquote}}{}
\IfFileExists{microtype.sty}{% use microtype if available
  \usepackage[]{microtype}
  \UseMicrotypeSet[protrusion]{basicmath} % disable protrusion for tt fonts
}{}
\makeatletter
\@ifundefined{KOMAClassName}{% if non-KOMA class
  \IfFileExists{parskip.sty}{%
    \usepackage{parskip}
  }{% else
    \setlength{\parindent}{0pt}
    \setlength{\parskip}{6pt plus 2pt minus 1pt}}
}{% if KOMA class
  \KOMAoptions{parskip=half}}
\makeatother
\usepackage{xcolor}
\newif\ifbibliography
\setlength{\emergencystretch}{3em} % prevent overfull lines
\providecommand{\tightlist}{%
  \setlength{\itemsep}{0pt}\setlength{\parskip}{0pt}}
\setcounter{secnumdepth}{-\maxdimen} % remove section numbering
% definitions for citeproc citations
\NewDocumentCommand\citeproctext{}{}
\NewDocumentCommand\citeproc{mm}{%
  \begingroup\def\citeproctext{#2}\cite{#1}\endgroup}
\makeatletter
 % allow citations to break across lines
 \let\@cite@ofmt\@firstofone
 % avoid brackets around text for \cite:
 \def\@biblabel#1{}
 \def\@cite#1#2{{#1\if@tempswa , #2\fi}}
\makeatother
\newlength{\cslhangindent}
\setlength{\cslhangindent}{1.5em}
\newlength{\csllabelwidth}
\setlength{\csllabelwidth}{3em}
\newenvironment{CSLReferences}[2] % #1 hanging-indent, #2 entry-spacing
 {\begin{list}{}{%
  \setlength{\itemindent}{0pt}
  \setlength{\leftmargin}{0pt}
  \setlength{\parsep}{0pt}
  % turn on hanging indent if param 1 is 1
  \ifodd #1
   \setlength{\leftmargin}{\cslhangindent}
   \setlength{\itemindent}{-1\cslhangindent}
  \fi
  % set entry spacing
  \setlength{\itemsep}{#2\baselineskip}}}
 {\end{list}}
\usepackage{calc}
\newcommand{\CSLBlock}[1]{\hfill\break\parbox[t]{\linewidth}{\strut\ignorespaces#1\strut}}
\newcommand{\CSLLeftMargin}[1]{\parbox[t]{\csllabelwidth}{\strut#1\strut}}
\newcommand{\CSLRightInline}[1]{\parbox[t]{\linewidth - \csllabelwidth}{\strut#1\strut}}
\newcommand{\CSLIndent}[1]{\hspace{\cslhangindent}#1}
\title[Joint Confidence Regions for Rankings]{Joint Confidence Regions for Rankings based on Correlated Estimates}
\author[Matala]{Matala, Shaine Rosewel}
\institute[UP Stat]{University of the Philippines}
\usepackage{algorithm}
\usepackage{algpseudocode}
\usepackage{pgfpages}
\setbeameroption{show notes on second screen=right}
\ifLuaTeX
  \usepackage{selnolig}  % disable illegal ligatures
\fi
\usepackage{bookmark}
\IfFileExists{xurl.sty}{\usepackage{xurl}}{} % add URL line breaks if available
\urlstyle{same}
\hypersetup{
  pdfauthor={Matala, Shaine Rosewel},
  hidelinks,
  pdfcreator={LaTeX via pandoc}}

\author{Matala, Shaine Rosewel}
\date{November 26, 2025}

\begin{document}

\begin{frame}
\maketitle
\end{frame}

\begin{frame}{Background of the Study}
\phantomsection\label{background-of-the-study}
\begin{itemize}[<+->]
\tightlist
\item
  In the problem of estimating ranks of several unknown real-valued
  parameters \(\theta_1,\theta_2,\ldots, \theta_K\), it is desired to
  rank these \(K\) parameters from smallest to largest,
  \(\theta_{(1)}<\theta_{(2)}<\ldots<\theta_{(K)}\).
\item
  Given estimates in a table without an explicit ranking, readers tend
  to compare units by looking for smallest or largest estimates and for
  relative standings between the units. Such tables as motivate
  ``implicit'' rankings.
\item
  Because rankings based on the observed values of
  \(\hat\theta_1,\hat\theta_2,\ldots, \hat\theta_K\) can vary because of
  sampling variability, widely understood statements of uncertainty
  should accompany each released ranking.
\item
  The margin of error gives uncertainty in the estimate \(\hat\theta_k\)
  for each unit separately
\item
  A direct assessment of the uncertainty in the estimated overall
  ranking would simultaneously involve all units and their relative
  standing to each other
\end{itemize}
\end{frame}

\begin{frame}{Background of the Study}
\phantomsection\label{background-of-the-study-1}
\begin{itemize}[<+->]
\tightlist
\item
  Let \(r_1,r_2,\ldots,r_K\) be the true unknown ranks of
  \(\theta_1,\theta_2,\ldots, \theta_K\). A mathematical definition of
  \(r_k\) is as follows: \begin{equation}
  r_k = \sum^K_{j=1} I(\theta_j \leq \theta_k) = 1 + \sum_{j:j \neq k} I(\theta_j \leq \theta_k), \qquad \text{for} \; k = 1, 2, \dots, K.
  \label{eq:rank1}
  \end{equation}
\item
  Goal

  \begin{enumerate}[<+->]
  \tightlist
  \item
    derive a joint confidence region for
    \(\theta_1,\theta_2,\ldots, \theta_K\)
  \item
    obtain joint confidence intervals for the \(r_1,r_2,\ldots,r_K\)
    using a result from Klein et al. (2020)).
  \end{enumerate}
\end{itemize}
\end{frame}

\begin{frame}{Klein's}
\phantomsection\label{kleins}
Suppose that for each \(k \in \left\{1, 2, \dots, K\right\}\) there
exists values \(L_k\) and \(U_k\) ST

\begin{equation}
  \theta_k \in \left( L_k, U_k \right), k=1,2,\ldots,K.
  \label{eq:theta_int}
\end{equation}

If the condition in (\ref{eq:theta_int}) holds, the main result from
Klein et al. (2020) gives a range for the value of \(r_k\) for each
\(k \in \left\{1, 2, \dots, K\right\}\) as follows:

\begin{equation}
  r_k \in 
  \left\{ 
  \lvert \Lambda_{Lk} \rvert + 1,  
  \lvert \Lambda_{Lk} \rvert + 2,
  \lvert \Lambda_{Lk} \rvert + 3,
  \dots,
  \lvert \Lambda_{Lk} \rvert + \lvert \Lambda_{Ok} \rvert + 1
  \right\}
  \label{eq:klein_jcs}
\end{equation}
\end{frame}

\begin{frame}{Klein's}
\phantomsection\label{kleins-1}
Suppose that for random quantities \(L_k\) and \(U_k\) the event defined
in (\ref{eq:theta_int}) satisfies the following probability condition:

\begin{equation}
  P\left[ \bigcap^K_{k=1} \left\{ \theta_k \in \left(L_k, U_k\right) \right\} \right] \geq 1-\alpha,
  \label{eq:joint_cov1}
\end{equation} then, by the result of Klein et al. (2020), it also
follows that

\begin{equation}
  P\left[
  \bigcap^K_{k=1}
  \left\{
  r_k \in 
  \left\{ 
  \lvert \Lambda_{Lk} \rvert + 1,  
  \lvert \Lambda_{Lk} \rvert + 2,
  \dots,
  \lvert \Lambda_{Lk} \rvert + \lvert \Lambda_{Ok} \rvert + 1
  \right\}
  \right\}
  \right] \geq 1-\alpha.
  \label{eq:joint_cov2}
\end{equation}
\end{frame}

\begin{frame}{Other related studies}
\phantomsection\label{other-related-studies}
\end{frame}

\begin{frame}{Motivation}
\phantomsection\label{motivation}
\begin{itemize}[<+->]
\tightlist
\item
  Problem: Assuming independence when constructing joint confidence
  regions for estimators that are, in fact, correlated may lead to
  overly conservative and thus wider intervals, implying greater
  uncertainty.
\item
  Aim: develop a procedure capable of handling such dependencies while
  maintaining coverage close to the nominal level and producing
  relatively narrow joint confidence intervals.
\end{itemize}
\end{frame}

\begin{frame}{Motivation}
\phantomsection\label{motivation-1}
\begin{block}{Political setting --- David \& Legara (2015)}
\phantomsection\label{political-setting-legara}
\begin{itemize}[<+->]
\tightlist
\item
  Name recall is a powerful predictor of likely victory in elections.
\item
  In weak-party systems, candidates who belong to the same political
  alliance or ticket commonly co-occur in ballots and hence perform with
  similarity.
\end{itemize}

\note{
\begin{itemize}
  \item Candidates with a name-recall advantage, such as media celebrities, incumbents, and members of dynastic families, received majority of the votes in the 2010 senatorial elections
  \item top-ranked candidates is composed of people who can take the most advantage of name recall: All belong to at least one of the following types: media celebrity, member of political dynasty, or had prior experience in the Senate (labeled henceforth Celebrities and Dynasties). Of the eight candidates, three are former movie and television actors, three are offspring of former presidents and senators, and six had prior experience in the Senate. They come from different political parties and different tickets
  \item This set of candidates was aggressively campaigned alongside candidate for President Aquino, who was popular
throughout the election season and eventually won by a 12% margin.This is the only ticket-based cluster;
\end{itemize}
}
\end{block}
\end{frame}

\begin{frame}{Motivation}
\phantomsection\label{motivation-2}
\begin{block}{Measurement across geographies --- Klein et al. (2020)}
\phantomsection\label{measurement-across-geographies-klein}
\begin{table}[h]
\centering
\begin{tabular}{p{1cm} p{4.5cm} p{4.5cm}}
\textbf{Travel Time} & \textbf{Population Density} & \textbf{Common Locations} \\ \hline
Shorter mean &
Large unpopulated land areas; fewer high-density population centers &
Mountain region and Central region states \\[3pt] %3pt is the gap%
Longer mean &
Highly urbanized areas with large populations and dense population centers &
East Coast states \\
\end{tabular}
\end{table}

\note{
\begin{itemize}
  \item @klein also noted that states with large unpopulated land areas and relatively few high-density population centers tend to report shorter travel times while longer travel times are typically observed in highly urbanized states with large populations and high population densities.
  \item 2019---Many states with shorter travel times are located in the Mountain and Central regions, whereas majority of those with longer travel times are concentrated along the East Coast.
\end{itemize}
}
\end{block}
\end{frame}

\begin{frame}{Objective}
\phantomsection\label{objective}
This research aims to do the following:

\begin{itemize}[<+->]
\tightlist
\item
  Develop a procedure to construct joint confidence intervals for the
  ranks and the ranked parameters when the estimates to be ranked may be
  correlated.
\item
  Evaluate the performance of the proposed approaches under various
  parameter settings.
\item
  Apply the proposed approaches to a real-life example.
\end{itemize}
\end{frame}

\begin{frame}{Definitions and Assumptions}
\phantomsection\label{definitions-and-assumptions}
\begin{itemize}[<+->]
\tightlist
\item
  Define
  \(\hat{\boldsymbol{\theta}}=(\hat{\theta}_1,\hat{\theta}_2,\ldots,\hat{\theta}_K)'\)
  and assume that
  \(\hat{\boldsymbol{\theta}}\sim N(\boldsymbol{\theta},\boldsymbol{\Sigma})\)
  where \(\boldsymbol{\theta}=(\theta_1,\theta_2,\ldots,\theta_K)'\) is
  unknown and \(\boldsymbol{\Sigma}\) is a known \(K \times K\) positive
  definite matrix. The diagonal elements of \(\boldsymbol{\Sigma}\) are
  \(\sigma^2_1,\ldots,\sigma^2_K\).\\
\item
  Note that in the literature on inferences on the ranks, it is
  customary to assume that the variances are known.
\end{itemize}
\end{frame}

\begin{frame}{Procedure}
\phantomsection\label{procedure}
\begin{enumerate}[<+->]
\tightlist
\item
  Derive simultaneous confidence intervals for
  \(\theta_1,\theta_2,\ldots,\theta_K\) of the form \begin{equation}
  \mathfrak{R}_1 = 
  [\hat{\theta}_1 \pm t \times \sigma_1] \times
  [\hat{\theta}_2 \pm t \times \sigma_2] \times
  \cdots \times
  [\hat{\theta}_K \pm t \times \sigma_K].
  \label{eq:rev1}
  \end{equation}
\end{enumerate}

\begin{enumerate}[<+->]
\setcounter{enumi}{1}
\tightlist
\item
  Once the confidence intervals in (\ref{eq:rev1}) have been obtained,
  we can then use the result of Klein et al. (2020) in
  (\ref{eq:joint_cov2}) to get the lower and upper bounds on the ranks
  \(r_k, k=1,2,\ldots,K\). That is, we also get a joint confidence
  region for \(r_1, r_2, . . . , r_K\).
\end{enumerate}
\end{frame}

\begin{frame}{Proposed methodology to compute the joint confidence
region for the unordered parameters: Algorithm 1}
\phantomsection\label{proposed-methodology-to-compute-the-joint-confidence-region-for-the-unordered-parameters-algorithm-1}
Let the data be represented by
\(\hat{\boldsymbol{\theta}} = \left( \hat \theta_1, \hat \theta_2, \dots, \hat \theta_K \right)'\)
and suppose that \(\boldsymbol{\Sigma}\) is known

\begin{algorithmic}[1]
        \For {$b = 1, 2, \dots, B$}
                \State Generate $\hat{\boldsymbol{\theta}}^*_b \sim N_K \left( \hat{\boldsymbol{\theta}}, \boldsymbol{\Sigma}\right)$ and write $\hat{\boldsymbol{\theta}}^*_b = \left( \hat\theta^*_{b1}, \hat\theta^*_{b2}, \dots, \hat\theta^*_{bK} \right)' $
                \State Compute 
                \Statex \begin{minipage}{\linewidth}
                \centering
                $t^*_b = \underset{1 \leq k \leq K}{\max} \Bigg| \frac{\hat\theta^*_{bk} - \hat\theta_{k}}{\sigma_k} \Bigg|$
                \end{minipage}
        \EndFor
        \State Compute the $\left(1-\alpha\right)$-sample quantile of $t^*_1, t^*_2, \dots, t^*_B$, call this $\hat{t}$.
        \State The joint confidence region for $\boldsymbol{\theta} = (\theta_1, \theta_2, \dots, \theta_K)'$ is given by 
        \Statex \begin{minipage}{\linewidth}
    \centering
$\mathfrak{R}_1 = \left[ \hat\theta_1 \pm \hat t \times \sigma_1  \right] \times \left[ \hat\theta_2 \pm \hat t \times \sigma_2  \right] \times \dots \times \left[ \hat\theta_K \pm \hat t \times \sigma_K  \right]$.
    \end{minipage}
    \end{algorithmic}
\end{frame}

\begin{frame}{Algorithm 1: Quantile Calculation}
\phantomsection\label{algorithm-1-quantile-calculation}
We want the joint confidence region in (\ref{eq:rev1}) to satisfy the
following probability condition:

\begin{equation}
 P\left( \hat{\theta}_k-t \cdot \sigma_k \leq  \theta_k \leq  \hat{\theta}_k + t \cdot \sigma_k, \,\, \forall\; k=1,2,\ldots,K \right)  =1-\alpha.
\end{equation}

Equivalently, we require

\begin{equation}
 P\left( \max_{k=1,2,\ldots,K} \left| \dfrac{\hat{\theta}_k-\theta_k}{\sigma_k} \right| \le t \right) =1-\alpha. 
\end{equation}
\end{frame}

\begin{frame}{Proposed methodology to compute a joint confidence region
for the ordered parameters}
\phantomsection\label{sec:nonrank}
\begin{minipage}{1.0\textwidth}
\begin{algorithmic}[1]
    \For {$b = 1, 2, \dots, B$}
        \State
            Generate $\hat{\boldsymbol{\theta}}^*_b = \left( \hat{\theta}^*_{b1}, \hat{\theta}^*_{b2}, \dots, \hat{\theta}^*_{bK} \right)' \sim N_K \left( \boldsymbol{\hat \theta}, \boldsymbol {\Sigma} \right)$ and let $\hat{\theta}^*_{b(1)}, \hat{\theta}^*_{b(2)}, \dots, \hat{\theta}^*_{b(K)}$ be the corresponding ordered values 
        \State
        Compute $\hat\sigma^*_{b(k)}$ using:
        \Statex \qquad • asymptotic variance definition
        \Statex \qquad • second-level bootstrap
        \State
            Compute
\Statex
\begin{center}
$t^*_b = \underset{1 \leq k \leq K}{\max} \Bigg| \frac{\hat\theta^*_{b(k)} - \hat\theta^*_{k}}{\hat\sigma^*_{b(k)}} \Bigg|$
\end{center}
    \EndFor
    \State Compute the $\left(1-\alpha\right)$-sample quantile of $t^*_1, \dots, t^*_B$, call this $\hat{t}$.
    \State The joint confidence region of $\theta_{(1)}, \theta_{(2)}, \dots, \theta_{(K)}$ is given by
\Statex
\begin{center}
    $\mathfrak{R}_2 = \left[ \hat\theta_{(1)} \pm \hat t \times \hat\sigma_{(1)}  \right] \times \left[ \hat\theta_{(2)} \pm \hat t \times \hat\sigma_{(2)}  \right] \times \dots \times \left[ \hat\theta_{(K)} \pm \hat t \times \hat\sigma_{(K)}  \right]$
\end{center}
    \end{algorithmic}
\end{minipage}
\end{frame}

\begin{frame}{Algorithm 2: Asymptotic Definition of Variance}
\phantomsection\label{algorithm-2-asymptotic-definition-of-variance}
\(\hat\sigma^*_{b(k)} = \sqrt{\left[\text{kth ordered value among} \ \left\{ \hat{\theta}^{*2}_{b1} + \sigma_1^2, \dots, \hat{\theta}^{*2}_{bK} + \sigma_K^2 \right\}\right] - \hat {\theta}^{*2}_{(k)}}\)
\end{frame}

\begin{frame}{Algorithm 3: Variance from Second-Level Bootstrap}
\phantomsection\label{algorithm-3-variance-from-second-level-bootstrap}
\begin{algorithmic}[1]
            \For {$c = 1, 2, \dots, C$}
                \State \begin{minipage}[t]{\dimexpr\linewidth-\algorithmicindent} Generate $\hat{\boldsymbol{\theta}}^{**}_{bc} = \left( \hat{\theta}^{**}_{bc1}, \hat{\theta}^{**}_{bc2}, \dots, \hat{\theta}^{**}_{bcK} \right) \sim N_K \left( \hat{\boldsymbol{\theta}}_b^*, \boldsymbol {\Sigma} \right)$ and let $\hat{\theta}^{**}_{bc(1)}, \hat{\theta}^{**}_{bc(2)}, \dots, \hat{\theta}^{**}_{bc(K)}$ be the corresponding ordered values of $\hat{\theta}^{**}_{bc1}, \hat{\theta}^{**}_{bc2}, \dots, \hat{\theta}^{**}_{bcK}$
                \end{minipage}
                \State \begin{minipage}[t]{\dimexpr\linewidth-\algorithmicindent} Compute
                $\displaystyle \hat{\sigma}^*_{b(k)} = \frac{\sum^C_{c=1} \left( \hat \theta^{**}_{bc(k)} - \bar {\hat\theta}^{**}_{b \cdot (k)} \right)^2}{C-1}, \quad \bar {\hat\theta}^{**}_{b\cdot(k)} = \frac{1}{C} \sum^C_{c=1} {\hat\theta}^{**}_{bc(k)}$
                \end{minipage}
                \EndFor
    \end{algorithmic}
\end{frame}

\begin{frame}{Evaluation Algorithm}
\phantomsection\label{evaluation-algorithm}
For given values of \(\boldsymbol{\Sigma}\) and
\(\theta_1, \theta_2, \dots, \theta_K\) (with corresponding
\(\theta_{(1)}, \theta_{(2)}, \dots, \theta_{(K)}\) for rank-based
methods)

\begin{algorithmic}[1] % Start algorithmic block
            \For {$\text{replications} = 1, 2, \dots, 5000$}
            \State Generate $\hat{\boldsymbol{\theta}} \sim N_K(\boldsymbol{\theta}, \boldsymbol{\Sigma})$
            \State Compute the confidence region $\mathfrak{R}_1$ for the unordered parameters using Algorithm 1 and the confidence region for the ordered parameters $\mathfrak{R}_2$ using Algorithms 2 and 3.
            \State For the unordered parameters, check if $\left( \theta_1, \theta_2, \dots, \theta_K\right) \in \mathfrak{R}_1$ and compute $T_1, T_2$, and $T_3$. For the ordered parameters, check if $\left( \theta_{(1)}, \theta_{(2)}, \dots, \theta_{(K)}\right) \in \mathfrak{R}_2$
        \EndFor
    \State Compute the proportion of times that the condition in line 4 is satisfied and the average of $T_1, T_2$, and $T_3$.
    \end{algorithmic}
\end{frame}

\begin{frame}{Covariance Matrix \(\boldsymbol{\Sigma}\)}
\phantomsection\label{covariance-matrix-boldsymbolsigma}
\begin{itemize}
\tightlist
\item
  The covariance matrix \(\boldsymbol{\Sigma}\) need not be a diagonal
  matrix.
\item
  We assume that \(V(\hat{\boldsymbol{\theta}})=\boldsymbol{\Sigma}\) is
  known and express \(\boldsymbol{\Sigma}\) as in
  (\ref{eq:sigma_matrix}), where \(\mathbf{R}\) is the population
  correlation matrix.
\end{itemize}

\begin{equation}
  \boldsymbol{\Sigma} = \boldsymbol{\Delta}^{1/2} \mathbf{R} \boldsymbol{\Delta}^{1/2}; \quad \boldsymbol{\Delta} = \text{diag} \left\{ \sigma^2_1, \sigma^2_2, \dots, \sigma^2_K \right\}.
  \label{eq:sigma_matrix}
\end{equation}

\begin{itemize}
\tightlist
\item
  The diagonal elements of \(\boldsymbol{\Sigma}\), which are
  \(\sigma^2_k=V(\hat{\theta}_k)\) for \(k =1,2, \ldots,K\), are treated
  as known quantities in practice.
\end{itemize}
\end{frame}

\begin{frame}{Correlation Structures: Equicorrelation}
\phantomsection\label{correlation-structures-equicorrelation}
\begin{itemize}[<+->]
\tightlist
\item
  We assume certain correlation structures among the
  \(\hat{\theta}\)s.\\
\item
  Equicorrelation is considered for simplicity.\\
\item
  This assumes that the \(k\) variables are equally correlated, i.e.,
  that \(\rho_{jk}=\rho\) where \(\rho \in [-1,1]\) for
  \(j \neq k \in \{1, \dots, K\}\). \begin{equation}
  \mathbf{R}_{\text{eq}} = \left( 1-\rho \right) \mathbf{I}_K + \rho \boldsymbol{1}_K \boldsymbol{1}'_K = 
  \begin{bmatrix}
  1 & \rho & \cdots & \rho \\
  \rho & 1 & \cdots & \rho \\
  \vdots & \vdots & \ddots & \vdots \\
  \rho & \rho & \cdots & 1
  \end{bmatrix}_{K \times K}
  \label{eq:equicorrelation}
  \end{equation}
\end{itemize}
\end{frame}

\begin{frame}{Correlation Structures: Block correlation}
\phantomsection\label{correlation-structures-block-correlation}
\begin{itemize}[<+->]
\tightlist
\item
  Useful in the context of pre-election surveys
\item
  In a block correlation matrix \(\mathbf{R}_{block}\) with \(G\)
  blocks, each diagonal block represents an equicorrelation structure
  within group \(g\), denoted by \begin{equation}
  \mathbf{R}_{\text{eq,g}} = \left( 1-\rho_{g} \right) \mathbf{I}_{n_g} + \rho_{g} \boldsymbol{1}_{n_g} \boldsymbol{1}'_{n_g} \notag
  \end{equation} where \(\rho_{g}\) is the within-block correlation and
  \(n_g\) is the number of variables in block \(g\) such that
  \(\sum_{g=1}^G n_g = K\).\\
\item
  The off-diagonal blocks capture between-block correlations,
  represented by
  \(\mathbf{C}_{g'g} = \mathbf{C}_{gg'} = \rho_{gg'}\boldsymbol{1}_{n_g} \boldsymbol{1}'_{n_g}\)
  where \(g\neq g' \in \{1, \dots, G\}\)
\item
  The full block correlation matrix can be expressed as in
  (\ref{eq:blockcorrelation}).\\
  \begin{equation}
  \mathbf{R}_{\text{block}} = 
  \begin{bmatrix}
  \mathbf{R}_{eq,1} & \mathbf{C}_{12} & \cdots & \mathbf{C}_{1G} \\
  \mathbf{C}_{21} & \mathbf{R}_{eq,2} & \cdots & \mathbf{C}_{2G} \\
  \vdots & \vdots & \ddots & \vdots \\
  \mathbf{C}_{G1} & \mathbf{C}_{G2} & \cdots & \mathbf{R}_{eq,G}
  \end{bmatrix}_{K \times K}
  \label{eq:blockcorrelation}
  \end{equation}
\end{itemize}
\end{frame}

\begin{frame}{Correlation Structures: Distance-based correlation}
\phantomsection\label{correlation-structures-distance-based-correlation}
\begin{itemize}[<+->]
\tightlist
\item
  Useful in the mean travel time from the study of Klein et al. (2020).
\item
  Spatial dependence can be modeled using a stationary Matérn
  correlation function, which for two locations \(\mathbf{s}_i\) and
  \(\mathbf{s}_j\) is expressed as \begin{equation}
  \rho_{\text{matern}} = \frac{2^{1-\nu}}{\Gamma(\nu)} (\kappa \;\Vert \;\mathbf{s}_i - \mathbf{s}_j \; \Vert)^\nu K_\nu  (\kappa \;\Vert \;\mathbf{s}_i - \mathbf{s}_j \; \Vert)
  \notag
  \end{equation} where \(\Vert \cdot \Vert\) denotes the Euclidean
  distance and \(K_\nu\) is the second kind of the modified Bessel
  function. It has a scale parameter \(\kappa > 0\) and a smoothness
  parameter \(\nu > 0\). \(\rho_{\text{matern}}\) reduces to the
  exponential correlation when \(\nu = 0.5\) and to Gaussian correlation
  function when \(\nu = \infty\).
\end{itemize}
\end{frame}

\begin{frame}{Code blocks}
\phantomsection\label{code-blocks}
\begin{alertblock}{Alert block}
$$
E = mc^2
$$
\end{alertblock}

\begin{examples}
Example blocks are automatically green in color
\end{examples}

\begin{block}{Blue block}
\phantomsection\label{blue-block}
\begin{itemize}
\tightlist
\item
  happens with level 2, 3 headings
\item
  this is only true for `Madrid' theme in R Markdown!!
\end{itemize}
\end{block}
\end{frame}

\begin{frame}{This works, incremental bullets}
\phantomsection\label{this-works-incremental-bullets}
\begin{itemize}[<+->]
\tightlist
\item
  Bullet 1
\item
  Bullet 2
\end{itemize}
\end{frame}

\begin{frame}{This nests, but does not increment}
\phantomsection\label{this-nests-but-does-not-increment}
\begin{itemize}
\tightlist
\item
  Bullet 1
\item
  Bullet 2

  \begin{itemize}
  \tightlist
  \item
    subbullet 1
  \item
    subbullet 2
  \end{itemize}
\end{itemize}
\end{frame}

\begin{frame}{This increments}
\phantomsection\label{this-increments}
\begin{itemize}[<+->]
\tightlist
\item
  Bullet 1
\item
  Bullet 2

  \begin{itemize}[<+->]
  \tightlist
  \item
    subbullet 1
  \item
    subbullet 2
  \end{itemize}
\end{itemize}
\end{frame}

\begin{frame}{This increments too}
\phantomsection\label{this-increments-too}
\begin{itemize}[<+->]
\tightlist
\item
  Bullet 1
\item
  Bullet 2

  \begin{itemize}[<+->]
  \tightlist
  \item
    subbullet 1
  \item
    subbullet 2
  \end{itemize}
\end{itemize}
\end{frame}

\begin{frame}{Algo Font Size Adjustments sample}
\phantomsection\label{algo-font-size-adjustments-sample}
\begin{minipage}{0.9\textwidth}
\begin{algorithmic}[1]
    \For {$b = 1, 2, \dots, B$}
        \State \parbox[t]{0.85\linewidth}{\footnotesize{
            Generate $\hat{\boldsymbol{\theta}}^*_b = \left( \hat{\theta}^*_{b1}, \hat{\theta}^*_{b2}, \dots, \hat{\theta}^*_{bK} \right)' \sim N_K \left( \boldsymbol{\hat \theta}, \boldsymbol {\Sigma} \right)$ and let $\hat{\theta}^*_{b(1)}, \hat{\theta}^*_{b(2)}, \dots, \hat{\theta}^*_{b(K)}$ be the corresponding ordered values 
        }}
        \State \parbox[t]{0.85\linewidth}{\footnotesize{
            Compute $\hat\sigma^*_{b(k)} = \hat\sigma^*_{b(k)} = \sqrt{\left[\text{kth ordered value among} \ \left\{ \hat{\theta}^{*2}_{b1} + \sigma_1^2, \dots, \hat{\theta}^{*2}_{bK} + \sigma_K^2 \right\}\right] - \hat {\theta}^{*2}_{(k)}}$}}
            \Statex
        \State \parbox[t]{0.85\linewidth}{\footnotesize{ 
            Compute}}
\Statex
\begin{center}
\parbox{0.9\linewidth}{
    \centering
    {\footnotesize
$t^*_b = \underset{1 \leq k \leq K}{\max} \Bigg| \frac{\hat\theta^*_{b(k)} - \hat\theta^*_{k}}{\hat\sigma^*_{b(k)}} \Bigg|$
    }}
\end{center}
    \EndFor
    \State \parbox[t]{0.85\linewidth}{\footnotesize{Compute the $\left(1-\alpha\right)$-sample quantile of $t^*_1, \dots, t^*_B$, call this $\hat{t}$.}}
    \State \parbox[t]{0.85\linewidth}{\footnotesize{The joint confidence region of $\theta_{(1)}, \theta_{(2)}, \dots, \theta_{(K)}$ is given by
     }}
\Statex
\begin{center}
\parbox{0.9\linewidth}{
    \centering
    {\footnotesize
    $\mathfrak{R}_2 = \left[ \hat\theta_{(1)} \pm \hat t \times \hat\sigma_{(1)}  \right] \times \left[ \hat\theta_{(2)} \pm \hat t \times \hat\sigma_{(2)}  \right] \times \dots \times \left[ \hat\theta_{(K)} \pm \hat t \times \hat\sigma_{(K)}  \right]$
    }}
\end{center}
\Statex
\parbox{0.9\linewidth}{
    {\footnotesize
    where
    }}
\Statex
\begin{center}
\parbox{0.9\linewidth}{
    \centering
    {\footnotesize
    $\hat\sigma_{(k)} =
    \sqrt{
      \text{kth ordered value among }
      \{ \hat\theta_1^{2} + \sigma_1^2,\,
         \dots,\,
         \hat\theta_K^{2} + \sigma_K^2 \}
      - \hat\theta_{(k)}^{2}
    }$
    }}
\end{center}
    \end{algorithmic}
\end{minipage}
\end{frame}

\begin{frame}{Text}
\phantomsection\label{text}
Today is going to be a great day

\begin{block}{Today is\ldots{}}
\phantomsection\label{today-is}
A new beginning.
\end{block}

\begin{block}{Your Reference}
\phantomsection\label{your-reference}
\begin{itemize}
\tightlist
\item
  \href{https://www.slideshare.net/slideshow/create-beamer-slide-with-r-markdown/105158435}{slideshare}
\item
  \href{https://mpetroff.net/files/beamer-theme-matrix/}{themes}
\item
  \href{https://stackoverflow.com/questions/49367078/incremental-sub-bullets-in-rmarkdown-and-beamer}{incremental
  bullets}
\end{itemize}

\phantomsection\label{refs}
\begin{CSLReferences}{1}{0}
\bibitem[\citeproctext]{ref-legara}
David, C., \& Legara, E. F. (2015). \emph{How voters combine candidates
on the ballot: The case of the philippine senatorial elections}.

\bibitem[\citeproctext]{ref-klein}
Klein, M., Wright, T., \& Wieczorek, J. (2020). A joint confidence
region for an overall ranking of populations. \emph{Journal of the Royal
Statistical Society}, 589--606.

\end{CSLReferences}
\end{block}
\end{frame}

\end{document}
